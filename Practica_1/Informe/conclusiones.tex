
\begin{itemize}
    \item El análisis del filtro estadístico de media móvil mostró que la selección del tamaño de la ventana de promediado es muy importante en el diseño. Si bien aumentar la cantidad de muestras atenúa de manera más eficiente la varianza del ruido Gaussiano, esto introduce inevitablemente un retardo temporal en la salida, lo cual representa un parámetro restrictivo en sistemas que operan en tiempo real.
    
    \item El acondicionamiento digital de señales mioelectrocas se confirma como una etapa crucial para el diseño y validación de sistemas de control de protesis electromecanicas. La estabilización de estas señales, mediante la reducción de ruido estadístico, es lo que garantiza que los actuadores mecánicos reciban comandos limpios, evitando oscilaciones o respuestas erráticas en la prótesis.
    \item En general, los resultados experimentales obtenidos confirman el comportamiento teórico de un diferenciador digital. La prueba con señal cuadrada permitió verificar la detección de cambios abruptos, mientras que la prueba con señal senoidal validó la relación matemática entre una función y su derivada. Esto demuestra que el bloque implementado cumple correctamente la función esperada dentro del entorno GNU Radio.
\end{itemize}


