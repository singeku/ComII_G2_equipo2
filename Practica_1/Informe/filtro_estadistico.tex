\subsection{Acondicionamiento Estadístico: Filtro de Media Móvil}

La presencia de ruido Gaussiano es un desafío inherente en la adquisición de señales físicas. Para mitigar las fluctuaciones aleatorias y mejorar la relación señal a ruido (SNR), se diseñó un bloque de procesamiento estadístico personalizado. La estrategia seleccionada corresponde a un filtro discreto de media móvil, regido por la siguiente ecuación de diferencias \cite{oppenheim}:

\begin{equation}
    y[n] = \frac{1}{N} \sum_{i=0}^{N-1} x[n-i] 
\end{equation}

Donde $N=10$ representa el tamaño de la ventana de promediado, $x[n]$ es la señal de entrada contaminada y $y[n]$ es la salida estabilizada.

\subsubsection{Implementación en GNU Radio}
La arquitectura del sistema se construyó utilizando el entorno de GNU Radio Companion. El filtro se programó desde cero mediante un \textit{Embedded Python Block}. Como se evidencia en el diagrama de flujo (Fig. \ref{fig:diagrama_gnu}), la señal original se somete a una fuente de ruido antes de ingresar al bloque diseñado, permitiendo evaluar su desempeño en tiempo real.

\begin{figure}[ht]
    \centering
    % RECUERDA CAMBIAR 'diagrama_bloques.png' POR EL NOMBRE EXACTO DE TU FOTO
    \includegraphics[width=\linewidth]{imagenes/filtro_estadistico.png}
    \caption{Diagrama de bloques implementado en GNU Radio, destacando el módulo de filtro estadístico en Python.}
    \label{fig:diagrama_gnu}
\end{figure}

\subsubsection{Validación y Aplicación Práctica}
La capacidad de programar filtros a medida es un requerimiento crítico en aplicaciones de alta complejidad, como el diseño y validación de sistemas de control de fuerza para la reproducción de movimientos básicos de la mano \cite{emg_analysis}. En este contexto, las señales mioeléctricas crudas presentan una alta varianza que puede desestabilizar a los actuadores mecánicos. Aplicar un promedio estadístico es el paso previo obligatorio para garantizar un control suave y preciso.

Al ejecutar el sistema, los resultados obtenidos (Fig. \ref{fig:grafica_filtro}) demuestran la eficacia del bloque. La estadística aplicada atenúa significativamente los picos del ruido Gaussiano. Aunque persiste un leve rizado debido a una ventana pequeña ($N=10$), este diseño garantiza una baja latencia computacional, un factor indispensable para el tiempo de respuesta inmediato que exige el control de una prótesis real.

\begin{figure}[ht]
    \centering
    % RECUERDA CAMBIAR 'grafica_senal.png' POR EL NOMBRE EXACTO DE TU FOTO
    \includegraphics[width=\linewidth]{imagenes/grafica_filtro_media_movil.png}
    \caption{Resultados del procesamiento: Comparación entre la señal contaminada con ruido Gaussiano y la señal estabilizada.}
    \label{fig:grafica_filtro}
\end{figure}