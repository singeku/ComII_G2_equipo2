\section{acondicionamiento de señal: filtro de media movil}

para mitigar el efecto del ruido gaussiano introducido en la señal original, se implemento un filtro estadistico en python integrado como un bloque de GNU radio. el diseño seleccionado corresponde a un filtro de media movil discreto, por la siguiente ecuacion de diferencias:

\begin{equation}
    y[n] = \frac{1}{n} \sum_{i=0}^{N-1} x[n-i]
\end{equation}

donde $N$ representa el tamaño de la ventana (numero de muestras promediadas, configurado en $N=10$), $x[n]$ es la señal de entrada contaminada y $y[n]$ es la salida suavizada

\subsection{analisis y aplicacion practica}
este tipo de acondicionamiento digital es critico no solo en telecomunicaciones, sino en el procesamiento de biopotenciales. por ejemplo, en el \textit{design and validation of a force control system in the reproduction of basic movements}, estabilizar señales erraticas (como las lecturas electromiograficas) mediante filtros similares es un paso fundamental antes de que la informacion ingrese al sistema de control.

\begin{figure}[ht]
    \centering
    \includegraphics[width=0.8\linewidth]{imagenes/grafica_filtro_media_movil.png}
    \caption{Comparación entre señal original (rojo) y señal filtrada (azul).}
    \label{fig:filtro}
\end{figure}