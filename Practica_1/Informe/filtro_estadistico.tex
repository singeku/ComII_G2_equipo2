\subsection{Acondicionamiento Estadístico: Filtro de Media Móvil}

El filtrado de ruido Gaussiano es un desafío inherente en la adquisición de señales físicas. Para mitigar las fluctuaciones aleatorias y mejorar la relación señal a ruido (SNR), se diseñó un bloque de procesamiento estadístico personalizado en GNU Radio mediante un \textit{Embedded Python Block}. La estrategia seleccionada correspondió a un filtro discreto de media móvil, regido por la siguiente ecuación de diferencias \cite{oppenheim}:

\begin{equation}
    y[n] = \frac{1}{N} \sum_{i=0}^{N-1} x[n-i] 
\end{equation}

Donde $N=10$ representó el tamaño de la ventana de promediado, $x[n]$ la señal de entrada contaminada y $y[n]$ la salida estabilizada.

\subsubsection{Validación y Aplicación Práctica}
La capacidad de programar filtros a medida fue un requerimiento crítico evaluado para aplicaciones de alta complejidad, tales como el diseño y validación de un sistema de control de fuerza en la reproducción de movimientos básicos de la mano \cite{emg_analysis}. En este contexto, dado que las bioseñales crudas presentan una alta varianza que desestabiliza los actuadores mecánicos, la aplicación de un promedio estadístico resultó ser un paso previo obligatorio para garantizar un control suave.

Para validar el bloque, la señal original se sometió a una fuente de ruido Gaussiano. El desempeño del filtro se sintetiza en la Tabla \ref{tab:filtro_resultados}.

\begin{table}[htbp]
\centering
\caption{Síntesis de validación del filtro de media móvil.}
\label{tab:filtro_resultados}
\begin{tabular}{|p{1.5cm}|p{2.5cm}|p{3.5cm}|}
\hline
\textbf{Condición} & \textbf{Efecto en la Salida} & \textbf{Conclusión} \\
\hline
Ruido Gaussiano & Atenuación de picos de alta frecuencia. & Mejoró la relación SNR estabilizando la señal base. \\
\hline
Ventana corta ($N=10$) & Leve rizado residual. & Minimizó la latencia computacional, vital para el control en tiempo real. \\
\hline
\end{tabular}
\end{table}

Al ejecutar el sistema, los resultados obtenidos corroboraron la eficacia del bloque. Tal como se observó en la Fig. \ref{fig:grafica_filtro}, la estadística aplicada atenuó significativamente las variaciones aleatorias. Aunque persistió un rizado debido al tamaño reducido de la ventana, este diseño garantizó la baja latencia exigida para la reproducción instantánea de movimientos, logrando un equilibrio óptimo entre suavidad de la señal y velocidad de respuesta.

\begin{figure}[htbp]
    \centering
    \includegraphics[width=\linewidth]{imagenes/grafica_filtro_media_movil.png}
    \vspace{-2mm}
    \caption{Comparación temporal entre la señal contaminada con ruido Gaussiano y la salida estabilizada por el filtro.}
    \label{fig:grafica_filtro}
\end{figure}