\subsection{Diferenciador Discreto}

El diferenciador discreto aproxima la derivada temporal de una señal midiendo la tasa de cambio entre muestras consecutivas. Para su implementación en GNU Radio, se programó un \textit{Embedded Python Block} basado en la ecuación de diferencias finitas $y[n] = x[n] - x[n-1]$. El sistema se configuró como un \texttt{sync\_block} de tipo \texttt{float32}, el cual requirió almacenar el valor de la muestra previa ($x[n-1]$) en la memoria interna del bloque para mantener la continuidad matemática durante el procesamiento por vectores que realiza el software.

\subsubsection{Validación y Aplicación Práctica}
El comportamiento del bloque se evaluó estimulando el sistema con un \textit{Signal Source} y visualizando el contraste entre la señal original y la derivada en un sumidero temporal. Se ejecutaron dos pruebas fundamentales para analizar la respuesta ante variaciones suaves y cambios abruptos. Los resultados teóricos y prácticos se sintetizan en la Tabla \ref{tab:dif_resultados}.

\begin{table}[htbp]
\centering
\caption{Síntesis de pruebas de validación del diferenciador discreto.}
\label{tab:dif_resultados}
\begin{tabular}{|p{1.5cm}|p{2.5cm}|p{3.5cm}|}
\hline
\textbf{Señal de Entrada} & \textbf{Forma de Salida} & \textbf{Conclusión del Comportamiento} \\
\hline
Senoidal & Cosenoidal. & Evidenció el desfase teórico de $\pi/2$ radianes, validando la derivación en señales trigonométricas continuas. \\
\hline
Cuadrada & Tren de impulsos. & Actuó como detector de flancos, generando picos en las transiciones y valores nulos en los tramos constantes. \\
\hline
\end{tabular}
\end{table}

El comportamiento descrito se corroboró visualmente en las gráficas de la Fig. \ref{fig:dif_pruebas}. Tal como se observó, la salida del sistema respondió fielmente a las transiciones de las señales de entrada. En la prueba de la onda cuadrada, los picos de amplitud confirmaron la elevada pendiente en los puntos de cambio brusco. Por otro lado, la prueba senoidal demostró la dependencia de la amplitud resultante frente a la frecuencia angular. Adicionalmente, se concluyó que, debido a la naturaleza de la discretización por diferencias finitas, la fidelidad de la aproximación dependió fuertemente de contar con una frecuencia de muestreo suficientemente alta para evitar irregularidades.

\begin{figure}[htbp]
    \centering
    \begin{subfigure}{0.48\linewidth}
        \includegraphics[width=\linewidth]{imagenes/Senoidal.jpeg}
        \caption{Prueba con señal senoidal.}
    \end{subfigure}
    \hfill
    \begin{subfigure}{0.48\linewidth}
        \includegraphics[width=\linewidth]{imagenes/Cuadrada.jpeg}
        \caption{Prueba con señal cuadrada.}
    \end{subfigure}
    \vspace{-2mm}
    \caption{Respuesta del diferenciador discreto ante señales de entrada continua y abrupta.}
    \label{fig:dif_pruebas}
\end{figure}