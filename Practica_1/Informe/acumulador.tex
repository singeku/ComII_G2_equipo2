\subsection{Acumulador discreto con memoria}

El acumulador discreto ideal, definido recursivamente como $y[n] = y[n-1] + x[n]$, requirió adaptación para su implementación en GNU Radio debido a su arquitectura de procesamiento por bloques de tamaño $N$. Dado que computar únicamente la suma acumulada local reiniciaba el valor en cada iteración de la función \texttt{work()}, se diseñó un \textit{Embedded Python Block} para preservar el estado del sistema. Matemáticamente, la salida del bloque se definió como $\mathbf{y} = A + \mathrm{cumsum}(\mathbf{x})$, donde el estado interno se actualizó guardando el último valor procesado ($A \leftarrow y_{N-1}$) como condición inicial del siguiente bloque.

\subsubsection{Validación del bloque}
El comportamiento del sistema se evaluó mediante un flujograma con un \textit{Vector Source} cíclico acoplado al bloque acumulador. Para comprobar la continuidad temporal, se ejecutaron tres pruebas fundamentales, cuyos resultados y conclusiones se sintetizan en la Tabla \ref{tab:acum_resultados}.

\begin{table}[htbp]
\centering
\caption{Síntesis de pruebas de validación del acumulador con memoria.}
\label{tab:acum_resultados}
\begin{tabular}{|p{1.5cm}|p{2.5cm}|p{3.5cm}|}
\hline
\textbf{Señal de Entrada} & \textbf{Forma de Salida} & \textbf{Conclusión del Comportamiento} \\
\hline
Constante ($x[n]=1$) & Rampa creciente. & Evidenció la deriva temporal del sistema ante señales con componente DC. \\
\hline
Media cero ($[1, 2, 5, -4, -4]$) & Señal acotada. & Validó la conservación de la memoria en régimen continuo sin desbordamientos. \\
\hline
Bipolar ($\pm 1$) & Onda triangular. & Comprobó el comportamiento del bloque como un integrador discreto ideal. \\
\hline
\end{tabular}
\end{table}

Las formas de onda resultantes de estas tres pruebas se graficaron en la Fig. \ref{fig:acum_pruebas}. Tal como se observó en las gráficas, el sistema mantuvo la coherencia del estado interno a lo largo del tiempo sin importar el tipo de entrada, lo que permitió concluir que la memoria compartida entre los bloques de GNU Radio operó correctamente sin pérdidas de datos.

\begin{figure}[htbp]
    \centering
    \begin{subfigure}{0.32\linewidth}
        \includegraphics[width=\linewidth]{imagenes/fig_acum_resultado_dc.png}
        \caption{Componente DC.}
    \end{subfigure}
    \hfill
    \begin{subfigure}{0.32\linewidth}
        \includegraphics[width=\linewidth]{imagenes/fig_acum_resultado.png}
        \caption{Media cero.}
    \end{subfigure}
    \hfill
    \begin{subfigure}{0.32\linewidth}
        \includegraphics[width=\linewidth]{imagenes/fig_acum_resultado_cuadrada.png}
        \caption{Bipolar ($\pm 1$).}
    \end{subfigure}
    \vspace{-2mm}
    \caption{Respuesta del acumulador con memoria ante diversas secuencias cíclicas.}
    \label{fig:acum_pruebas}
\end{figure}